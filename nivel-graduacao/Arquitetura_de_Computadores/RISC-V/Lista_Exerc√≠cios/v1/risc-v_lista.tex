% !TeX spellcheck = pt_BR
% ANUfinalexam.tex (Version 2.0)
% ===============================================================================
% Australian National University Final Exam LaTeX template.
% 2004; 2009, Timothy Kam, ANU School of Economics
% Licence type: Free as defined in the GNU General Public Licence: http://www.gnu.org/licenses/gpl.html

\documentclass[a4paper,12pt,fleqn]{article}
\usepackage{amsmath}
\usepackage{fancyhdr}
\usepackage{verbatim}

\usepackage[utf8]{inputenc} % Entrada do Teclado
\usepackage[T1]{fontenc}    % Saída para a compilação
\usepackage[brazil]{babel}  % Idioma de Compilação
\usepackage{url}

% Insert your course information here %%%%%%%%%%%%%%%%%%%%%%%%%%%%%%%%%%

\newcommand{\institution}{UNIVERSIDADE FEDERAL DE OURO PRETO}
\newcommand{\titlehd}{Arquitetura de Computadores}
\newcommand{\examtype}{Lista: RISC-V}
\newcommand{\examdate}{\today}
\newcommand{\examcode}{BCC 236}
\newcommand{\readtime}{Quinze Minutos}
\newcommand{\writetime}{Duas Horas}
\newcommand{\materials}{Nenhum Material Auxiliar é permitido}
\newcommand{\lastwords}{Fim do Teste}

%%%%%%%%%%%%%%%%%%%%%%%%%%%%%%%%%%%%%%%%%%%%%%%%%%%%

%\setcounter{MaxMatrixCols}{10}
\newtheorem{theorem}{Theorem}
\newtheorem{acknowledgement}[theorem]{Acknowledgement}
\newtheorem{algorithm}[theorem]{Algorithm}
\newtheorem{axiom}[theorem]{Axiom}
\newtheorem{case}[theorem]{Case}
\newtheorem{claim}[theorem]{Claim}
\newtheorem{conclusion}[theorem]{Conclusion}
\newtheorem{condition}[theorem]{Condition}
\newtheorem{conjecture}[theorem]{Conjecture}
\newtheorem{corollary}[theorem]{Corollary}
\newtheorem{criterion}[theorem]{Criterion}
\newtheorem{definition}[theorem]{Definition}
\newtheorem{example}[theorem]{Example}
\newtheorem{exercise}[theorem]{Exercise}
\newtheorem{lemma}[theorem]{Lemma}
\newtheorem{notation}[theorem]{Notation}
\newtheorem{problem}[theorem]{Problem}
\newtheorem{proposition}[theorem]{Proposition}
\newtheorem{remark}[theorem]{Remark}
\newtheorem{solution}[theorem]{Solution}
\newtheorem{summary}[theorem]{Summary}
\newenvironment{proof}[1][Proof]{\noindent\textbf{#1.} }{\ \rule{0.5em}{0.5em}}

% ANU Exams Office mandated margins and footer style
\setlength{\topmargin}{0cm}
\setlength{\textheight}{9.25in}
\setlength{\oddsidemargin}{0.0in}
\setlength{\evensidemargin}{0.0in}
\setlength{\textwidth}{16cm}
\pagestyle{fancy}
\lhead{} 
\chead{} 
\rhead{} 
\lfoot{} 
\cfoot{\footnotesize{Página \thepage \ de \pageref{finalpage} -- \titlehd \ (\examcode)}} 
\rfoot{} 

% DEPRECATED: ANU Exams Office mandated margins and footer style
%\setlength{\topmargin}{0cm}
%\setlength{\textheight}{9.25in}
%\setlength{\oddsidemargin}{0.0in}
%\setlength{\evensidemargin}{0.0in}
%\setlength{\textwidth}{16cm}
%\pagestyle{fancy}
%\lhead{} %left of the header
%\chead{} %center of the header
%\rhead{} %right of the header
%\lfoot{} %left of the footer
%\cfoot{} %center of the footer
%\rfoot{Page \ \thepage \ of \ \pageref{finalpage} \\
%       \texttt{\examcode}} %Print the page number in the right footer

\renewcommand{\headrulewidth}{0pt} %Do not print a rule below the header
\renewcommand{\footrulewidth}{0pt}


\begin{document}

% Title page

\begin{center}
%\vspace{5cm}
\large\textbf{\institution}
\end{center}

\begin{center}
\textit{ \examtype\ -- \examdate \\ Orientação: Rodolfo Labiapari - rodolfolabiapari@decom.ufop.br}
\end{center}

\begin{center}
\large\textbf{\titlehd\ - \examcode}
\end{center}
\vspace{0.3cm}

\begin{comment}

\begin{center}
\textit{Reading Time: \readtime}
\end{center}
\begin{center}
\textit{Writing Time:  \writetime}
\end{center}
\begin{center}
\textit{Permitted Materials: \materials}
\end{center}
\end{comment}

% End title page
\it

\begin{quote}
	\it
	Observações:
	
	\begin{itemize}
		\item Responda às questões abaixo de forma completa; 
		\item Todo o conteúdo escrito será avaliado;
		\item Plágios serão tidos como nota 0 entre ambos os envolvidos;
		\item Referencie de forma correta respostas obtidas de materiais de terceiros;
		\item A lista deverá ser entregue na data da próxima aula, dia 30/01/17, até às 23:59;
		\item Pode ser entregue tanto de forma digital (formato PDF), ou física;
		\item Transparências para estudos e consultas: \url{goo.gl/jzI3Ok} e \url{https://goo.gl/PpRPWk}. 
	\end{itemize}
	
	%Responda \textbf{todas} as questões abaixo. É esperado respostas sucintas mas completas. Respostas irrelevantes serão penalizadas e por isso assim, \textbf{todo conteúdo escrito será avaliado}.
\end{quote}

\vspace{0.3cm}


\paragraph{Questão 1} Dê exemplos de aparelhos que utilizam ISA RISC.

\paragraph{Questão 2} Explique o que é o projeto RISC-V.

\paragraph{Questão 3} RISC-V é utilizado só para uso acadêmico?

\paragraph{Questão 4} Quais são as extensões oficiais do RISC-V ditas em aula e qual o propósito de cada uma?

\paragraph{Questão 5} É possível adicionar minha própria extensão, cujo nome é \texttt{extensaoUFOP} ao projeto?

\paragraph{Questão 6} Quantos registradores o RISC-V possui e quais são eles? Esse número pode ser alterado?

\paragraph{Questão 7} Cite duas otimizações realizadas no projeto do RISC-V.

\paragraph{Questão 8} RISC-V possui operações de multiplicação com acesso direto à memória principal?

\paragraph{Questão 9} Qual é o propósito de utilizar a extensão C?

\paragraph{Questão 10} Existe alguma diferença entre os formatos de operações do RV32I, RV64I e a Extensão C?

\paragraph{Questão 11} RISC-V possui o condicional \texttt{if}? Como ele realiza suas instruções de condição?

%\paragraph{Questão 12} Quantos formatos de instruções existem no RISC-V e quais são eles?

\paragraph{Questão 12} Observando as posições dos registradores na tabela de formato de instruções, por que eles estão situados na mesma posição em todas os formatos?

\paragraph{Questão 13} Qual a diferença das instruções \texttt{jump} e \texttt{jump and link}?

\paragraph{Questão 14} O que é o Rocket-Chip e quantos estágios ele possui?

\paragraph{Questão 15} Explique sucintamente como Rocket-Chip é utilizado.

\begin{center}
\vspace{3cm}
------------ \textit{\lastwords} ------------
\end{center}


\label{finalpage}

\end{document}
